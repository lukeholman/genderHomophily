\documentclass[12pt,]{article}
\usepackage{lmodern}
\usepackage{amssymb,amsmath}
\usepackage{ifxetex,ifluatex}
\usepackage{fixltx2e} % provides \textsubscript
\ifnum 0\ifxetex 1\fi\ifluatex 1\fi=0 % if pdftex
  \usepackage[T1]{fontenc}
  \usepackage[utf8]{inputenc}
\else % if luatex or xelatex
  \ifxetex
    \usepackage{mathspec}
  \else
    \usepackage{fontspec}
  \fi
  \defaultfontfeatures{Ligatures=TeX,Scale=MatchLowercase}
\fi
% use upquote if available, for straight quotes in verbatim environments
\IfFileExists{upquote.sty}{\usepackage{upquote}}{}
% use microtype if available
\IfFileExists{microtype.sty}{%
\usepackage{microtype}
\UseMicrotypeSet[protrusion]{basicmath} % disable protrusion for tt fonts
}{}
\usepackage[margin=1in]{geometry}
\usepackage{hyperref}
\hypersetup{unicode=true,
            pdftitle={Researchers preferentially collaborate with same-gendered colleagues across the life sciences},
            pdfauthor={Luke Holman* and Claire Morandin; *;},
            pdfborder={0 0 0},
            breaklinks=true}
\urlstyle{same}  % don't use monospace font for urls
\usepackage{graphicx,grffile}
\makeatletter
\def\maxwidth{\ifdim\Gin@nat@width>\linewidth\linewidth\else\Gin@nat@width\fi}
\def\maxheight{\ifdim\Gin@nat@height>\textheight\textheight\else\Gin@nat@height\fi}
\makeatother
% Scale images if necessary, so that they will not overflow the page
% margins by default, and it is still possible to overwrite the defaults
% using explicit options in \includegraphics[width, height, ...]{}
\setkeys{Gin}{width=\maxwidth,height=\maxheight,keepaspectratio}
\IfFileExists{parskip.sty}{%
\usepackage{parskip}
}{% else
\setlength{\parindent}{0pt}
\setlength{\parskip}{6pt plus 2pt minus 1pt}
}
\setlength{\emergencystretch}{3em}  % prevent overfull lines
\providecommand{\tightlist}{%
  \setlength{\itemsep}{0pt}\setlength{\parskip}{0pt}}
\setcounter{secnumdepth}{0}
% Redefines (sub)paragraphs to behave more like sections
\ifx\paragraph\undefined\else
\let\oldparagraph\paragraph
\renewcommand{\paragraph}[1]{\oldparagraph{#1}\mbox{}}
\fi
\ifx\subparagraph\undefined\else
\let\oldsubparagraph\subparagraph
\renewcommand{\subparagraph}[1]{\oldsubparagraph{#1}\mbox{}}
\fi

%%% Use protect on footnotes to avoid problems with footnotes in titles
\let\rmarkdownfootnote\footnote%
\def\footnote{\protect\rmarkdownfootnote}

%%% Change title format to be more compact
\usepackage{titling}

% Create subtitle command for use in maketitle
\newcommand{\subtitle}[1]{
  \posttitle{
    \begin{center}\large#1\end{center}
    }
}

\setlength{\droptitle}{-2em}
  \title{Researchers preferentially collaborate with same-gendered colleagues
across the life sciences}
  \pretitle{\vspace{\droptitle}\centering\huge}
  \posttitle{\par}
  \author{Luke Holman* and Claire Morandin\textsuperscript{§} \\ *\textit{luke.holman@unimelb.edu.au} \\ \textsuperscript{§}\textit{claire.morandin@helsinki.fi} \vspace{5mm}}
  \preauthor{\centering\large\emph}
  \postauthor{\par}
  \date{}
  \predate{}\postdate{}

\usepackage{microtype} \usepackage[all]{hypcap} \usepackage{amsmath}
\usepackage{booktabs} \usepackage{caption}
\usepackage[labelfont=bf]{caption} \usepackage{titling}
\pretitle{\begin{flushleft}\LARGE}
\posttitle{\par\end{flushleft}\vskip 0.5em}
\preauthor{\begin{flushleft}\large \lineskip 0.5em}
\postauthor{\par\end{flushleft}} \predate{\begin{flushleft}\large}
\postdate{\par\end{flushleft}} \usepackage{fancyhdr} \pagestyle{fancy}
\fancyhead[LO,LE]{\textsl{\leftmark}} \rhead[]{Gender and coauthorship}
\usepackage[noabbrev,capitalise]{cleveref} \usepackage{titlefoot}
\usepackage{amssymb} \usepackage{rotating} \usepackage{lineno}
\linenumbers

\begin{document}
\maketitle
\begin{abstract}
Evidence suggests that women in academia are hindered by conscious and
unconscious biases, and often feel excluded from formal and informal
opportunities for research collaboration. In addition to ensuring
fairness and helping to redress gender imbalance in the academic
workforce, increasing women's access to collaboration could help
scientific progress by drawing on more of the available human capital.
Here, we test whether researchers preferentially collaborate with
same-gendered colleagues, using more stringent methods and a larger
dataset than in past work. Our results reaffirm that researchers
preferentially co-publish with colleagues of the same gender, and show
that this `gender homophily' is slightly stronger today than it was 10
years ago. Contrary to our expectations, we found no evidence that
homophily is driven mostly by senior academics, and no evidence that
homophily is stronger in fields where women are in the minority.
Interestingly, journals with a high impact factor for their discipline
tended to have comparatively low homophily, as predicted if mixed-gender
teams produce better research. We discuss some potential causes of
gender homophily in science. \vspace{5mm}
\par\noindent \textbf{Keywords:} Gender bias, Homophily, Research
collaboration, Text mining, Women in STEM.
\end{abstract}

\maketitle

\unmarkedfntext{*School of BioSciences, The University of Melbourne, Victoria, Australia.}
\maketitle\unmarkedfntext{\textsuperscript{§}Organismal and Evolutionary Biology Research Programme,
Faculty of Biological and Environmental Sciences, University of Helsinki, Finland.}

\newpage

\section{Introduction}\label{introduction}

Women are severely underrepresented in many branches of science,
technology, engineering, mathematics, and medicine (STEMM), and face
additional challenges and inequities relative to men {[}1--5{]}. On
average, women occupy more junior positions {[}6,7{]} with lower
salaries {[}8,9{]}, receive less grant money {[}10,11{]}, are promoted
more slowly {[}12--15{]}, and are allocated fewer resources {[}16{]} and
less research funding {[}17--19{]}. Experimental evidence suggests that
bias against women plays a major role in generating these differences
{[}20,21{]}.

Writing papers, networking, and collaboration are all instrumental to
research productivity and academic career advancement {[}22--25{]}, and
dozens of studies have tested for gender differences in these areas
{[}5,26--29{]}. For example, studies have concluded that women tend to
be less involved in international collaboration {[}19,28,30--32{]},
collaborate less within their own university departments {[}31{]}, have
less prestigious collaborations {[}33{]}, and fewer collaborations in
total {[}34{]}. These gender differences in collaboration practice
presumably have multiple causes, which might include implicit and
explicit gender bias {[}20{]}, differential family obligations
{[}33,35,36{]}, gender differences in confidence or self-esteem
{[}37{]}, concerns relating to sexual harassment {[}38{]}, and unequal
access to conferences {[}39{]} or travel funds {[}32{]}.

A high, steadily increasing proportion of research papers is written by
more than one author {[}3{]}, making collaboration a key predictor of
publication output, and thus of career prospects {[}40,41{]}.
Additionally, empirical studies imply that mixed-gender or otherwise
diverse teams produce better outputs on collaborative tasks than less
diverse teams {[}42--48{]}. For reasons such as these, multiple studies
have examined the author lists of published research articles in order
to test for gender differences in collaboration frequency or pattern. To
our knowledge, most or all such studies imply that men co-publish with
men, and women with women, more often than expected if collaborators
assort randomly with respect to gender {[}49--58{]}. This pattern of
assortative publishing has often been termed `gender homophily'.

However, we believe that prior studies of gender homophily were hindered
by a largely unacknowledged statistical issue that we name the Wahlund
effect (\autoref{wahlund_plot}), by analogy with the conceptually
similar Whalund effect in population genetics {[}59{]}. The Wahlund
effect makes it deceptively difficult to infer gender-based preferences
simply by counting the number of same- and mixed-gender coauthorships.
Essentially, whenever coauthorship data are sampled from two or more
discrete sets of literature, which vary in the author gender ratio and
which are largely not connected by collaboration, the number of
same-gendered coauthors will be inflated. This can give the impression
that authors preferentially publish with same-gendered colleagues even
if no gender preferences exist, or if the true preference is for
opposite-gendered colleagues (`gender heterophily'). For example, a
sample of literature containing bioinformatics and cell biology papers
will probably contain an excess of mostly-male and mostly-female author
lists, simply because researchers usually collaborate within their own
discipline, and because the author gender ratio is more male-biased in
bioinformatics than in cell biology {[}5{]}.

\begin{figure}[htbp]
\centering
\includegraphics{../figures/Fig1.pdf}
\caption{The Wahlund effect can make it appear as if authors prefer to
publish with same-gendered colleagues, even if no such preference
exists. Here, coloured circles represent male and female authors, and
coauthors are linked with lines. Across the whole set of ten papers,
there is an apparent excess of same-gender collaborations: there are six
same-gender papers and only four mixed-gender papers, which is fewer
than the \(10\times2\times0.5\times0.5 = 5\) mixed-gender papers
expected under the null hypothesis that authors assort randomly with
respect to gender. However, within each subset, there is no evidence
that authors prefer to publish with same-gendered individuals (if
anything, this small dataset suggests gender heterophily). The Wahlund
effect will tend to inflate the frequency of same-sex coauthorships
whenever the data is composed of two or more disconnected subsets of
literature with different author gender ratios; these subsets could be
research disciplines, older versus newer papers, or papers from authors
in different countries. The example countries and disciplines were
selected based on {[}5{]}. \label{wahlund_plot}}
\end{figure}

In the present study, we test whether life sciences researchers tend to
co-publish with same-gendered colleagues, while controlling for the
Wahlund effect as strictly as possible. We use a recently-published
dataset describing the gender of 35.5m authors from 9.15m articles
indexed on PubMed {[}5{]}. Holman et al. {[}5{]} reported large
differences in the gender ratio of authors across research disciplines,
journals, countries, and across the years 2002-2016. We therefore tested
for gender homophily while restricting our analysis to particular
journals (i.e.~research specialties), time periods, and countries. We
quantified gender assortment using a metric called \(\alpha'\) {[}60{]},
which is positive when same-gender authors publish together more often
than expected (gender homohily), negative when opposite-gender authors
publish together more often than expected (heterophily), and equal to
zero when authors assort randomly with respect to gender (see Methods).

\section{Results}\label{results}

\subsection{Gender homophily by discipline, time period, and authorship
position}\label{gender-homophily-by-discipline-time-period-and-authorship-position}

\autoref{alpha_histograms} shows the distribution of \(\alpha'\)
estimates in 2015-2016 across all journals for which we recovered
sufficient data, when \(\alpha'\) was calculated for all authors, first
authors only, or last authors only. Most journals had positive values of
\(\alpha'\) (77-92\%, depending on time period and author type; S1
Data), and for many of these the FDR-corrected p-values suggested that
\(\alpha'\) was significantly greater than zero (1469/2077 journals were
significant in 2015-16, and 404/1192 in 2005-6; S1 Data). Only 2/2077
journals had statistically significantly heterophily (i.e.
\(\alpha' < 0\)) in 2015-16, and 1/1192 in 2005-6 (S2 Table). The
remaining 606 or 787 journals (in 2015 and 2005 respectively) had a
value of \(\alpha'\) not significantly different from zero, consistent
with the null hypothesis of random assortment with respect to gender. We
also confirmed that the majority of papers had multiple authors, in most
journals (S2 Data) and disciplines (S3 Data, S1 Fig).

\begin{figure}[htbp]
\centering
\includegraphics{../figures/Fig2.pdf}
\caption{Of the 2116 journals for which we had adequate data in
2015-2016, 825 showed statistically significant evidence of gender
homophily (denoted by \(\alpha' > 0\)), and 1 showed statistically
significant evidence of heterophily (\(\alpha' < 0\)), after false
discovery rate correction. The white area shows the number of journals
for which homophily was significantly stronger than expected under the
null hypothesis (corrected p \textless{} 0.05), while the blue area
shows all the remainder. Patterns were similar whether \(\alpha'\) was
calculated for all authors, for first authors only, or for last authors
only. \label{alpha_histograms}}
\end{figure}

\(\alpha'\) was significantly higher in the literature sample from
2015-16 relative to 2005-6, though the difference in means was small (S2
Fig; Effect of the fixed factor `Time period' in a linear mixed model of
the data for all author positions: Cohen's \(d\) = \(0.091{\pm}0.04\),
\(t_{953}\) = 2.42, p = 0.016).

When comparing pairs of \(\alpha'\) values estimated for the first and
last authors for the same journals, we found that \(\alpha'\) tended to
be higher for first authors than for last authors (S3 Fig; Effect of the
fixed factor `Authorship position' in a linear mixed model: Cohen's
\(d\) = \(0.065{\pm}0.02\), \(t_{2024}\) = 4.28, p \textless{} 0.0001).
This suggests that the gender of the first author was a slightly
stronger predictor of the remaining authors' genders than the gender of
the last author, i.e.~the opposite of what is predicted if senior
scientists are causally responsible for homophily.

\subsection{Variance in homophily between
disciplines}\label{variance-in-homophily-between-disciplines}

\autoref{alpha_histograms} illustrates the variance in journal homophily
values (\(\alpha'\)) across scientific disciplines. All disciplines had
positive mean \(\alpha'\), although homophily appeared somewhat stronger
in some disciplines than others (e.g.~mean \(\alpha'\) was
\(0.12{\pm}0.02\) for Urology journals and \(0.03{\pm}0.01\) for
Veterinary Medicine journals; \autoref{alpha_histograms}, S4 Data).
However, there was no formal evidence for consistent differences in
\(\alpha'\) between disciplines: the random factor `Discipline'
explained around 1\% of the variance in \(\alpha'\) in the two linear
mixed models described in the previous section (see
\autoref{alpha_histograms} and mixed models in Online Supplementary
Material). Thus, the processes responsible for producing positive
\(\alpha'\) values appear to be similarly strong in all the disciplines
we examined.

There was no indication that journals publishing on a wide range of
topics have higher \(\alpha'\) values than more specialised journals,
due to the Wahlund effect. For example, the journal category
`Multidisciplinary' -- which includes journals like \emph{PLoS ONE},
\emph{Nature}, \emph{Science}, and \emph{PNAS} -- did not have notably
elevated \(\alpha'\) (\autoref{alpha_histograms}). This result suggests
that our estimates of homophily, and estimates from some earlier
studies, are not notably inflated by the presence of disparate research
topics (with variable author gender ratios) being published within
individual journals.

\subsection{Relationship between gender homophily and number of
authors}\label{relationship-between-gender-homophily-and-number-of-authors}

Papers with two authors had significantly lower (but still positive)
\(\alpha'\) values relative to papers with more than two authors, on
average (\autoref{author_number}; statistical results in Online
Supplmentary Material). Papers with 3, 4 or 5+ authors had essentially
identical average \(\alpha'\) values. The variance in \(\alpha'\) across
journals was also a little higher for 2-authors papers compared to the
remainder (\autoref{author_number}), though part of this variance is due
to the reduced sample size (in terms of number of authors) for the
2-author papers.

\begin{figure}[htbp]
\centering
\includegraphics{../figures/Fig3.pdf}
\caption{The coefficient of homophily (\(\alpha'\)) was slightly less
positive when calculated for two-author papers only, relative to papers
with longer author lists. The individual points, whose distribution is
summarised by the violin plots, correspond to individual journals. The
larger white points show the mean for each group (and its 95\% CIs), as
calculated by a Bayesian meta-regression model accounting for repeated
measures of \(\alpha'\) within journals, as well as the precision with
which \(\alpha'\) was estimated. \label{author_number}}
\end{figure}

\subsection{Relationship between gender homophily and gender
ratio}\label{relationship-between-gender-homophily-and-gender-ratio}

We next tested whether researchers are more or less likely to publish
with same-gendered colleagues in strongly gender-biased disciplines
(e.g.~Surgery or Nursing), relative to disciplines with a comparatively
gender-balanced workforce (e.g.~Psychiatry). We found a positive,
non-linear relationship between the overall gender ratio of all authors
publishing in a particular journal {[}5{]}, and the estimated value of
\(\alpha'\) for all authors and for first authors
(\autoref{alpha_gender_ratio}). Journals with a balanced or
female-biased author gender ratio tended to have higher \(\alpha'\) than
journals with a male-biased author gender ratio (GAM smooth terms p
\textless{} 0.001; Online Supplementary Material). The relationship was
not statistically significant when \(\alpha'\) was calculated for last
authors (GAM, p = 0.142), though the trend appeared similar
(\autoref{alpha_gender_ratio}).

\begin{figure}[htbp]
\centering
\includegraphics{../figures/Fig4.pdf}
\caption{There is a weakly positive, non-linear relationship between the
gender ratio of authors publishing in a journal, and the coefficient of
homophily (\(\alpha'\)). Specifically, journals with 50\% women authors
or higher tended to have more same-sex coauthorships than did journals
with predominantly men authors. This relationship held whether
\(\alpha'\) was calculated for all authors, first authors only, or last
authors only. A negative value on the x-axis denotes an excess of men
authors, a positive value denotes an excess of women authors, and zero
denotes gender parity. The lines were fitted using generalised additive
models with the smoothing parameter \(k\) set to 3.
\label{alpha_gender_ratio}}
\end{figure}

\subsection{Relationship between journal impact factor and gender
homophily}\label{relationship-between-journal-impact-factor-and-gender-homophily}

We observed a noisy but statistically significant linear relationship
between standardised journal impact factor and \(\alpha'\), such that
journals with a high impact factor for their discipline had weaker
gender homophily than did journals with a low impact factor for their
discipline (\autoref{impact_factor}; linear regression: \(R^2\) = 0.043,
\(t_{1415}\) = -8.0, p \textless{} 0.0001). The slope of the regression
was \(-0.012{\pm}0.0015\), indicating that increasing the
discipline-standardised impact factor by one standard deviation is
associated with a reduction in \(\alpha'\) of 0.012.

\begin{figure}[htbp]
\centering
\includegraphics{../figures/Fig5.pdf}
\caption{Journal impact factor (expressed relative to the average for
the discipline) is negatively correlated with \(\alpha'\). The
relationship is noisy (\(R^2\) = 0.043), but the results suggest that
journals with strong homophily tend to have lower impact factors than
journals with weak homophily in the same discipline.
\label{impact_factor}}
\end{figure}

\subsection{Analysis accounting for differences in author gender ratio
between
countries}\label{analysis-accounting-for-differences-in-author-gender-ratio-between-countries}

When we restricted the analysis by country, we observed statistically
significant homophily for 72 of the 325 journal-country combinations
tested (64 unique journals and 18 unique countries), and no significant
heterophily (S4-S5 Fig). Additionally, the values of \(\alpha'\)
calculated for each journal-country combination were only very slightly
lower than the \(\alpha'\) values calculated for the journal as a whole
(i.e.~when pooling papers from different countries, as was done to make
\autoref{alpha_histograms}): on average, the difference in \(\alpha'\)
was only 0.002 (S6 Fig). These results suggest that our findings of
widespread homophily in the main analysis were not driven solely by a
Wahlund effect resulting from gender differences between countries.

\subsection{\texorpdfstring{Theoretical expectations for \(\alpha\) when
the gender ratio differs between career
stages}{Theoretical expectations for \textbackslash{}alpha when the gender ratio differs between career stages}}\label{theoretical-expectations-for-alpha-when-the-gender-ratio-differs-between-career-stages}

As shown in \autoref{simulation}, we predict that \(\alpha\) is expected
to be non-zero, even if collaborators are randomly selected with respect
to gender, provided that there is a gender gap between career stages.
The extent to which \(\alpha\) deviates from zero depends on the
relative frequencies of collaboration within and between career stages.
When \textgreater{}50\% of collaborations were between early and
established researchers, we expect gender heterophily (\(\alpha < 0\)).
Conversely, when \textgreater{}50\% of collaborations occured within
career stages, we expect gender homophily (\(\alpha > 0\)). In a few
parameter spaces (shown in red; \autoref{simulation}), \(\alpha\) was
quite high, and overlapped with the values that we estimated
(\autoref{alpha_histograms}).

\begin{figure}[htbp]
\centering
\includegraphics{../figures/Fig6_inkscape.pdf}
\caption{When there is a difference in gender ratio between early-career
and established researchers, and collaboration is non-random with
respect to career stage, the null expectation for \(\alpha\) deviates
from zero. An excess of collaborations between career stages gives the
appearance of gender heterophily (lower rows, blue areas), while an
excess of within-career stage collaborations produced apparent gender
homophily (upper rows, red areas). However, the conditions required for
strong gender homophily are quite restrictive, making it unlikely that
this issue explains all of the homophily observed in Figure 2. Contour
lines denote increments of 0.1. \label{simulation}}
\end{figure}

Despite this overlap, \autoref{simulation} suggests that our main
conclusions (and those of other studies of gender homophily) are
probably robust to this career stage issue. We only expect strongly
positive \(\alpha\) when A) the gender ratio is highly skewed across
career stages (e.g.~a 5-fold difference), and B) collaborations between
early and established researchers are very rare (e.g. \textless{}10\% of
the total). Both of these conditions are untrue for most fields: the
gender gap across careers stages is generally less pronounced {[}1,5{]},
and it is very common for early-career researchers to co-publish with an
established mentor {[}61{]}. However, one can get \(\alpha > 0\) for
realistic combinations of parameters, e.g.~a moderate shortage of women
in senior positions coupled with a moderate excess of within-career
stage collaboration, suggesting this effect might contribute to some of
the observed homophily (in this and previous studies).

Lastly, we note that if there is a gender gap between career stages and
coauthorships between early-career and established researchers comprise
\textgreater{}50\% of the total, then the baseline expectation for
\(\alpha\) is actually less than zero (blue areas in
\autoref{simulation}). Therefore, our results might under-estimate the
extent to which researchers preferentially select same-gendered
collaborators in some cases.

\section{Discussion}\label{discussion}

We found evidence that researchers preferentially publish with
same-gendered coauthors, even after implementing stringent controls for
Wahlund effects (\autoref{wahlund_plot}). Our study therefore reaffirms
earlier studies' conclusions {[}49--57,62{]} and establishes their
generality across the life sciences. Relatively few journals had
\(\alpha'\) values below zero, and almost no journals showed
statistically significant gender heterophily after controlling for
multiple testing. The excess of same-gender coauthorships was quite
large: many journals had \(\alpha' > 0.1\), indicating that the gender
ratio of men's and women's coauthors differs by \textgreater{}10\% in
absolute terms. In relative terms, our findings are even more striking:
for example, if men have 20\% female coauthors and women have 30\% (i.e.
\(\alpha' = 0.1\) in a field with a typical gender ratio {[}5{]}), then
women publish with women 50\% more often than men do.

An important limitation of our study is that we cannot reliably
determine the cause(s) of the observed excess of same-gender
coauthorships. As well as the obvious interpretation -- conscious or
unconcious selection of same-gender collaborators by men, women, or both
-- our results could be partly explained by uncontrolled Wahlund
effects. However, we suspect the contribution of these to be minor, for
four reasons: we found positive \(\alpha'\) after controlling for three
obvious sources of Wahlund effect; there was no inflation of \(\alpha'\)
in highly multidisciplinary journals; restricting the data by country
yielded similar estimates of \(\alpha'\); and we showed that differences
in gender ratio between career stages are unlikely to fully explain our
results. On balance, we believe the data suggest that it is likely that
some researchers do preferentially select same-gendered collaborators,
although the frequency and strength of this preference is difficult to
ascertain.

We hypothesised that disciplines with a strongly skewed gender ratio
might show the strongest gender homophily, e.g.~because being in the
minority might increase motivation to seek out same-gendered colleagues.
Contrary to this hypothesis, we found no evidence that gender homophily
is restricted to particular disciplines: \(\alpha'\) was similarly high
across the board (\autoref{alpha_histograms}). Interestingly, gender
homophily was weakest for journals with a male-biased author gender
ratio, and strongest in journals with a female-biased author gender
ratio. This may suggest that men are more likely to preferentially seek
out male collaborators in fields where men are a minority, relative to
the homophily displayed by women in fields where women are a minority.
However, this latter result is only tentatively supported since our
sample contains few journals in which most authors are women
(\autoref{alpha_gender_ratio}).

We also found that gender homophily was marginally stronger in 2015-2016
relative to 2005-2006. Although this trend might reflect a change in the
gender preferences of researchers seeking collaborators, there are
alternative (and perhaps more likely) explanations. For example, this
trend might result from the increasing number of women working in senior
positions in STEMM over the past decade {[}63--65{]}. As shown in
\autoref{simulation}, if enough coauthorships are between junior and
senior researchers, a large gender gap between career stages can give
the appearance of heterophily. As this gender gap between career stages
lessens, the observed values of \(\alpha'\) may increase.

Regarding our finding of weaker homophily among 2-author papers, we
suspect that many 2-author teams comprise a student/postdoc and a senior
staff member, making these teams especially likely to be mixed-gender,
due to the elevated gender gap among senior researchers {[}1,5{]}.
Assuming this interpretation is correct, this result suggests that our
reported \(\alpha'\) values may underestimate the strength of peoples'
preferences for same-gendered collaborators; essentially, women seeking
a senior collaborator could be constrained to work mostly with men,
meaning that people's ideal and realised gender preferences would be
mismatched. On a related note, Ghiasi et al. {[}51{]} argue that women
in engineering are ``compliant {[}in reproducing{]} male-dominated
scientific structures'' because they do not collaborate often enough
with other women (their Figure 7 suggests that coauthorships between
women are 30\% more frequent than expected under random assortment). By
contrast, we feel that it is not helpful to recommend that women
collaborate primarily with other women, e.g.~because this constrains
women's options and may be counter-productive (particularly in fields
like engineering, where 90\% of professors are men {[}1{]}). Instead, we
suggest that researchers of both genders can help to close the gender
gap in STEMM. In the context of collaboration, one way to do this is to
undertake self-examination to ensure that one is not inadvertently
overlooking or excluding female potential students and colleagues. One
should also take care to treat male and female collaborators equally,
e.g.~in terms of training and mentoring, allocation of work, and how one
frames or promotes the collaboration (e.g.~in conference presentations
or on a website); evidence suggests that unconscious bias causes people
to undervalue women's research achievements {[}20{]}, and possibly to
assign menial or under-valued tasks to women and more prestigious tasks
to men {[}61{]}.

Our study begs two questions: what causes gender homophily in science,
and are our results cause for concern? These questions are closely
related. For example, some of the homophily we observed might be caused
by women seeking to avoid harassment or sexism from men {[}38{]}, which
would clearly be concerning. Additionally, Sheltzer and Smith {[}66{]}
concluded that `elite' male academics (defined as recipients of major
honours) have a higher proportion of male students and postdocs than
non-elite male academics. This finding could contribute to the homophily
we observed, and is cause for concern since Sheltzer and Smith
{[}66{]}'s results might reflect discrimination against women during
hiring {[}20{]}, or avoidance by women of elite research groups
(e.g.~due to gender differences in confidence, or a perception that some
groups are sexist). We also found a little evidence that gender
homophily is detrimental to research quality, in that high-impact
journals tended to have weaker homophily. Assuming that papers published
in high-impact journals are of higher average quality {[}67{]}, our
results provide non-experimental support for the hypothesis that
mixed-gender teams produce better research than single-gender teams
{[}42--48{]}. Another issue is that if many collaborations are between
established researchers, there will be an excess of male-male
collaborations in fields where women in senior positions are rare; some
of the observed homophily might therefore reflect the elevated gender
gap among senior researchers.

On the other hand, homophily might have more benign causes.
Collaboration is often most enjoyable and productive when working with
like-minded people, who might be same-gendered more often than not. We
also suppose that some people consciously choose to preferentially
collaborate with women in order to help close the gender gap in the
workforce; this would create homophily if women do this more than men.
In support of this interpretation, women appear more likely than men to
promote the work of female colleagues by inviting them to give talks
{[}68,69{]}. Given that many collaborative research projects
unfortunately involve a gendered division of labour {[}61{]}, working
with a same-gendered colleague may provide exposure to new parts of the
research process, and (especially for the minority gender) a welcome
change of pace.

\section{Methods}\label{methods}

\subsection{The dataset}\label{the-dataset}

We used the dataset of PubMed author lists from Holman et al. {[}5{]}.
Briefly, that dataset was created by downloading every article indexed
on PubMed and attempting to infer each author's gender from their given
name. Each journal was assigned to one of 107 scientific disciplines,
using PubMed's journal categorisations in the interests of objectivity.
Because the present study focuses on co-authorship, all single-author
papers were discarded. We also discarded all papers for which we could
not determine the gender of every author with \({\ge}95\%\) certainty,
in order to simplify the statistical analysis. To mitigate Wahlund
effects caused by variation in the gender ratio of researchers over time
(see below), we also discarded all papers except those that were
published either 0-1 or 10-11 years before the PubMed data were
collected (i.e.~20\textsuperscript{th} August 2016). Lastly, we excluded
journals with fewer than 50 suitable papers. Detailed sample size
information is given in S1 Table.

\subsection{\texorpdfstring{Calculating \(\alpha\), the coefficient of
homophily}{Calculating \textbackslash{}alpha, the coefficient of homophily}}\label{calculating-alpha-the-coefficient-of-homophily}

Following Bergstrom et al. {[}60{]}, we defined the coefficient of
homophily as \(\alpha = p - q\), where \(p\) is the probability that a
randomly-chosen co-author of a \emph{male} author is a man and \(q\) is
the probability that a randomly-chosen co-author of a \emph{female}
author is a man. Like the Wahlund effect, \(\alpha\) is borrowed from
population genetics; for a set of 2-author papers, it is equivalent to
Wright's coefficient of inbreeding {[}70{]}. Mathematical work
illustrates that \(\alpha\) is closely related to alternative
network-based methods for quantifying homophily {[}71{]}.

To estimate \(\alpha\) for a particular subset of the scientific
literature, we estimated \(p\) as the average proportion of men's
co-authors who are men (averaged across all papers with at least one man
author), and \(q\) as the average proportion of women's co-authors who
are men (averaged across all papers with at least one woman author). To
estimate the 95\% confidence intervals on \(\alpha\) for a given set of
\(n\) papers, we sampled \(n\) papers with replacement 1000 times,
estimated \(\alpha\) on each sample, and recorded the 95\% quantiles of
the resulting 1000 estimates.

As well as calculating \(\alpha\) for all authors, we calculated
\(\alpha\) for first or last authors only. \(\alpha\) was again defined
as \(p - q\), but this time \(p\) was estimated as the average
proportion of male co-authors on papers with a male first (or last)
author, and \(q\) was estimated as the average proportion of male
co-authors on papers with female first (or last) authors. We did not
calculate \(\alpha\) for other authorship positions (e.g.~second or
third authors) because this would necessitate culling the dataset to
include only papers with a sufficiently long author list, complicating
interpretation of the results.

We also calculated \(\alpha\) for papers with 2, 3, 4 or \({\ge}5\)
authors, for all journals that had at least 50 suitable papers from
2015-2016 with the specified author list length.

Our test assumes that the expected value of \(\alpha\) is zero if
authors randomly assort, but for small datasets this assumption is not
always true (as pointed out by Carl Bergstrom in a blog post,
\url{http://www.eigenfactor.org/gender/assortativity/note_to_eisen.rtf}).
To borrow Prof.~Bergstrom's example, consider a small research specialty
comprising just two men and two women researchers, who have together
produced six two-author papers: one in each of the six possible
two-author combinations. For these six papers,
\(\alpha = -\frac{1}{3}\), even though same- and opposite-gendered
coauthors were selected in equal proportion to their frequency in the
pool of possible collaborators.

To control for the fact that the null expectation for \(\alpha\) is not
zero for small datasets, we devised an adjusted version of the
coefficient of homophily, which we term \(\alpha'\). Every time we
calculated \(\alpha\) for a set of papers, we also determined the
expected value of \(\alpha\) under the null hypothesis that authors
assort randomly with respect to gender. This was accomplished by
randomly permuting authors across papers 1000 times, recalculating
\(\alpha\), and taking the median. We then calculated \(\alpha'\) by
subtracting the null expectation for \(\alpha\) from the observed value.
We also used the null-simulated \(\alpha\) values to calculate a
two-tailed p-value for the observed value of \(\alpha\); the p-value was
defined as the proportion of null simulations for which
\(|\alpha_{null}| > |\alpha_{obs}|\). We applied false discovery rate
(FDR) correction to each set of p-values to account for multiple testing
{[}72{]}.

As expected, \(\alpha'\) was usually almost identical to \(\alpha\) (S7
Fig), but \(\alpha\) was downwardly biased relative to \(\alpha'\) for
small datasets (S8 Fig). Additionally, the correlation between
\(\alpha'\) and sample size was negligible (\(R^2 < 0.01\)), suggesting
that our calculation of \(\alpha'\) effectively removed the dependence
of \(\alpha\) on sample size. We therefore used \(\alpha'\) in all
analyses.

\subsection{Minimising the Wahlund effect: research discipline and time
period}\label{minimising-the-wahlund-effect-research-discipline-and-time-period}

To minimise bias in \(\alpha'\) due to the Wahlund effect, we restricted
each set of papers to a single research specialty to the greatest extent
allowed by our data. Specifically, we only calculated \(\alpha'\) for
individual journals, since papers from the same journal typically focus
on closely related topics. Although some journals, e.g. \emph{PLoS ONE},
publish research from diverse disciplines with very different author
gender ratios {[}5{]}, calculating \(\alpha'\) for these highly
multidisciplinary journals is still useful as a contrast. The difference
in \(\alpha'\) between highly multidisciplinary and more specialised
journals, e.g. \emph{PLoS ONE} versus \emph{PLoS Computational Biology},
gives an estimate of the extent to which multidisciplinarity inflates
\(\alpha'\).

As well as varying between disciplines, the gender ratio of authors has
changed markedly over time {[}5{]}. Because the gender ratio was more
male-biased in the past, \(\alpha'\) would be inflated if we calculated
it for a sample of papers published over a long enough time frame. To
minimise this effect, we only sampled papers from two one-year periods
(namely 2005-6 and 2015-16). The median change per year in \% (fe)male
authors across journals is below 0.5\% {[}5{]}, and so restricting our
dataset to a single year should prevent temporal changes in gender ratio
from noticeably affecting our estimates of \(\alpha'\).

\subsection{Minimising the Wahlund effect: author country of
affiliation}\label{minimising-the-wahlund-effect-author-country-of-affiliation}

A Wahlund effect could arise even if one calculates \(\alpha'\) for a
single discipline and time period, because of variation in the gender
ratio of researchers from different countries. For example, Holman et
al. {[}5{]} showed that PubMed-indexed authors based in Serbia are more
than twice as likely to be women as are authors based in Japan.
Therefore, a dataset containing a mix of papers from teams of authors
based in these two countries would contain an excess of same-sex
coauthorships, even if collaboration were random with respect to gender
within each country.

To address this issue, we also analysed every combination of journal and
author country of affiliation for which we had enough data (i.e.~50 or
more papers published in 2015-16). For simplicity, we restricted the
dataset to only include papers for which Holman et al. {[}5{]} had
identified the country of affiliation for all authors on the paper, and
all authors shared the same country of affiliation. Restricting the
dataset in this fashion produced enough data to measure \(\alpha'\) for
325 combinations of journal and country (median: 70 papers and 273
authors per combination).

\subsection{Calculating standardised journal impact
factor}\label{calculating-standardised-journal-impact-factor}

We obtained the 3-year impact factor for each journal from Clarivate
Analytics. To account for large differences in impact factor between
disciplines, we took the the residuals from a model with \(Log_{10}\)
impact factor as the response and the research discipline of the journal
as a random effect. Thus, journals with a positive standardised impact
factor have a higher mean number of citations than the average for
journals in their discipline. We then used Spearman rank correlation to
test whether \(\alpha'\) was correlated with impact factor across
journals.

\subsection{Statistical analysis}\label{statistical-analysis}

Previous authors {[}66,73{]} have hypothesised that senior scientists
preferentially recruit staff and students of the same gender, and/or
that junior researchers preferentially select same-gendered mentors. In
the majority of PubMed-indexed disciplines, authorship conventions mean
that the first-listed author is often an early-career researcher, while
the author listed last is more likely to be a senior researcher leading
a research team {[}74{]}. Assuming that senior researchers are the main
drivers of homophily and that there are enough papers with three or more
authors, we predict that the last author's gender will be the strongest
predictor of the remaining authors' genders (i.e.~the gender of the last
author will be more salient than that of the first author, or any other
authorship position). This is because the first author's gender would
simply be an imperfect correlate of the true causal effect, while the
last author's gender would be the causal effect itself.

To test whether \(\alpha'\) for last authors tends to be higher than
\(\alpha'\) for first authors for any given dataset, we used a linear
mixed model implemented in the \texttt{lme4} and \texttt{lmerTest}
packages for R, with \emph{authorship position} (first or last) as a
fixed factor, and \emph{journal} and \emph{research discipline} as
crossed random effects. The response variable was \(\alpha'\), and we
weighted each observation by the inverse of the standard error from our
estimate of \(\alpha'\), meaning that more accurate measurements of
\(\alpha'\) had more influence on the results. We used a similar model
to test for a difference in \(\alpha'\) between the 2005-6 and the
2015-16 datasets, with two differences: we fit year range as a two-level
fixed factor (instead of authorship position), and we used \(\alpha'\)
estimated for all authors (not first/last authors) as the response
variable.

The relationship between the gender ratio of authors publishing in a
journal and its \(\alpha'\) value appeared nonlinear (see Results). We
therefore fit a generalised additive model with thin plate regression
spline smoothing, implemented using the \texttt{mgcv} package for R.

To model the relationship between \(\alpha'\) and the number of authors
on the paper, we used a meta-regression model implemented in the R
package \texttt{brms} {[}75{]}. The model incorporated the standard
error associated with easch estimate of \(\alpha'\), had author number
as a fixed effect, and journal as a random intercept (to control for
repeated measures of each journal). We also fit a random slope of author
number within journal, thereby allowing the response to author number to
vary between journals. We used the default (weak) priors. The full
output of this model can be viewed in the Online Supplementary Material.

\subsection{\texorpdfstring{Theoretical expectations for \(\alpha\) when
the gender ratio differs between career
stages}{Theoretical expectations for \textbackslash{}alpha when the gender ratio differs between career stages}}\label{theoretical-expectations-for-alpha-when-the-gender-ratio-differs-between-career-stages-1}

In many STEMM subjects, the gender ratio is more skewed among
established researchers relative to early-career researchers {[}1,5{]}.
We hypothesised that this skew could potentially create both Wahlund
effects and `reverse' Wahlund effects. For example, imagine that the
majority of collaborations are between students and professors, and that
the gender ratio differs between career stages: we will then see an
excess of mixed-gender coauthorships (heterophily, \(\alpha < 0\)), even
if gender has no direct, causal effect. Similarly, a hypothetical field
in which students work only with students, and professors with
professors, would have apparent gender homophily (\(\alpha > 0\)).

We can think of no tractable method of controlling for this issue using
our dataset, which contains no information on career stage. Therefore,
we instead decided to derive the theoretical expectations for \(\alpha\)
when there is a difference in gender ratio across career stages, in
order to determine if and how this effect should alter our inferences.
For simplicity, our calculations assume there are only two career
stages, though we intuit that the general conclusions would also apply
to a multi-tier career ladder. Under the null model that gender has no
causal effect on collaboration, we calculated \(\alpha\) for various
combinations of the four free parameters, i.e.~the gender ratios for
early- and late-career researchers, and the relative frequency of
collaborations between early-early, early-late, and late-late
collaborations. We then used the theoretical expectations for \(\alpha\)
to qualify our main conclusions (see Results). The Online Supplementary
Material gives annotated R code used to derive the theoretical
expectations.

\subsection{Data availability and
reproducibility}\label{data-availability-and-reproducibility}

The Online Supplementary Material contains R scripts used to produce all
results, figures and tables; it can be viewed online at
\url{https://lukeholman.github.io/genderHomophily/}. The input data from
Holman et al. {[}5{]} is archived at \url{https://osf.io/bt9ya/}.

\section{Acknowledgements}\label{acknowledgements}

CM was supported by the Academy of Finland (284666 to the Centre of
Excellence in Biological Interactions). We thank Devi Stuart-Fox and
Dominique Potvin for helpful discussion.

\section{References}\label{references}

\hypertarget{refs}{}
\hypertarget{ref-Shaw_2012}{}
1. Shaw AK, Stanton DE. Leaks in the pipeline: separating demographic
inertia from ongoing gender differences in academia. Proceedings of the
Royal Society of London B. 2012;272: 3736--3741.

\hypertarget{ref-Lariviere_2013}{}
2. Larivière V, Ni C, Gingras Y, Cronin B, Sugimoto CR. Bibliometrics:
global gender disparities in science. Nature. 2013;504: 211--213.

\hypertarget{ref-West_2013}{}
3. West JD, Jacquet J, King MM, Correll SJ, Bergstrom CT. The role of
gender in scholarly authorship. PLoS ONE. 2013;8: e66212.

\hypertarget{ref-Elsevier_report}{}
4. Elsevier Report. Gender in the global research landscape.
elseviercom/research-intelligence/resource-library/gender-report. 2017;

\hypertarget{ref-Holman_2018}{}
5. Holman L, Stuart Fox D, Hauser CE. The gender gap in science: How
long until women are equally represented? PLoS Biology. 2018;16:
e2004956.

\hypertarget{ref-Wutte_2007}{}
6. Wutte M. Closing the gender gap. Nature. 2007;448: NJ101--NJ102.

\hypertarget{ref-Reuben_2014}{}
7. Reuben E, Sapienza P, Zingales L. How stereotypes impair women's
careers in science. Proceedings of the National Academy of Sciences.
2014;111: 4403--4408.

\hypertarget{ref-Trower_2002}{}
8. Trower CA, Chait RP. Faculty diversity: Why women and minorities are
underrepresented in the professoriate, and fresh ideas to induce needed
reform. Harvard Magazine. 2002;104: 33--37.

\hypertarget{ref-Umbach_2007}{}
9. Umbach PD. Gender equity in the academic labor market: An analysis of
academic disciplines. Research in Higher Education. 2007;48: 169--192.

\hypertarget{ref-Hosek_2005}{}
10. Hosek S, Cox AG, Ghosh-Dastidar B, Kofner A, Ramphal N, Scott J, et
al. Gender differences in major federal external grant programs. RAND
Corporation. 2005;

\hypertarget{ref-pohlhaus_2011}{}
11. Pohlhaus JR, Jiang H, Wagner RM, Schaffer WT, Pinn VW. Sex
differences in application, success, and funding rates for NIH
extramural programs. Academic Medicine. 2011;86: 759.

\hypertarget{ref-Zuckerman_1987}{}
12. Zuckerman H. Persistence and change in the careers of men and women
scientists and engineers. National Academy Press. 1987; 127--156.

\hypertarget{ref-Rosenfeld_1991}{}
13. Rosenfeld RA. Outcome analysis of academic careers. Review prepared
for the Office of Scientific and Engineering Personnel, National
Research Council. 1991;

\hypertarget{ref-Long_1993}{}
14. Long JS, Paul DA, Robert M. Rank advancement in academic careers:
Sex differences and the effects of productivity. American Sociological
Review. 1993; 703--722.

\hypertarget{ref-Hopkins_2013}{}
15. Hopkins AL, Jawitz JW, McCarty C, Goldman A, Basu NB. Disparities in
publication patterns by gender, race and ethnicity based on a survey of
a random sample of authors. Scientometrics. 2013;96: 515--534.

\hypertarget{ref-ODorchai_2009}{}
16. O'Dorchai S, Meulders D, Crippa F, Margherita A. She figures
2009--Statistics and indicators on gender equality in science.
Publications Office of the European Union. 2009;

\hypertarget{ref-Feldt_1986}{}
17. Feldt B. The faculty cohort study: School of medicine. Ann Arbor,
MI: Office of Affirmative Action. 1986;

\hypertarget{ref-Stack_2004}{}
18. Stack S. Gender, children and research productivity. Scientometrics.
2004;45: 891--920.

\hypertarget{ref-Lariviere_2011}{}
19. Larivière V, Vignola-Gagné E, Villeneuve C, Gélinas P, Gingras Y.
Sex differences in research funding, productivity and impact: an
analysis of Québec university professors. Scientometrics. 2011;87:
483--498.

\hypertarget{ref-Moss_2012}{}
20. Moss-Racusin CA, Dovidio JF, Brescoll VL, Graham MJ, Handelsman J.
Science faculty's subtle gender biases favor male students. Proceedings
of the National Academy of Sciences. 2012;109: 16474--16479.

\hypertarget{ref-Knobloch_2013}{}
21. Knobloch-Westerwick S, Glynn CJ, Huge M. Science faculty's subtle
gender biases favor male students. Science Communication. 2013;35:
603--625.

\hypertarget{ref-Lee_2005}{}
22. Lee S, Bozeman B. The impact of research collaboration on scientific
productivity. Social Studies of Science. 2005;35: 673--702.

\hypertarget{ref-Wuchty_2007}{}
23. Wuchty S, Jones BF, Uzzi B. The increasing dominance of teams in
production of knowledge. Science. 2007;316: 1036--1039.

\hypertarget{ref-Abramo_2009}{}
24. Abramo G, D'Angelo CA, Di Costa F. Research collaboration and
productivity: is there correlation? Higher Education. 2009;57: 155--171.

\hypertarget{ref-Lariviere_2015}{}
25. Larivière V, Gingras Y, Sugimoto CR, Tsou A. Team size matters:
Collaboration and scientific impact since 1900. Journal of the
Association for Information Science and Technology. 2015;66: 1323--1332.

\hypertarget{ref-Long_1992}{}
26. Long JS. Measures of sex differences in scientific productivity.
Social Forces. 1992;71: 159--178.

\hypertarget{ref-Bozeman_2011}{}
27. Bozeman B, Monica G. How do men and women differ in research
collaborations? An analysis of the collaborative motives and strategies
of academic researchers. Research Policy. 2011;40: 1393--1402.

\hypertarget{ref-Abramo_2013}{}
28. Abramo G, D'Angelo CA, Di Costa F. Gender differences in research
collaboration. Journal of Informetrics. 2013;7: 811--822.

\hypertarget{ref-Badar_2013}{}
29. Badar K, Hite JM, Badir YF. Examining the relationship of
co-authorship network centrality and gender on academic research
performance: The case of chemistry researchers in pakistan.
Scientometrics. 2013;94: 755--775.

\hypertarget{ref-Lewison_2001}{}
30. Lewison G. The quantity and quality of female researchers: A
bibliometric study of Iceland. Scientometrics. 2001;52: 29--43.

\hypertarget{ref-Webster_2001}{}
31. Webster BM. Polish women in science: A bibliometric analysis of
Polish science and its publications. Research Evaluation. 2001;10:
185--194.

\hypertarget{ref-Bozeman_2004}{}
32. Bozeman B, Corley E. Scientists' collaboration strategies:
implications for scientific and technical human capital. Research
Policy. 2004;33: 599--616.

\hypertarget{ref-Long_1990}{}
33. Long JS. The origins of sex differences in science. Social Forces.
1990;68: 1297--1316.

\hypertarget{ref-Fuchs_2001}{}
34. Fuchs S, Von Stebut J, Allmendinger J. Gender, science, and
scientific organizations in Germany. Minerva. 2001;39: 175--201.

\hypertarget{ref-Reskin_1978}{}
35. Reskin BF. Scientific productivity, sex, and location in the
institution of science. American Journal of Sociology. 1978;83:
1235--1243.

\hypertarget{ref-Wright_2003}{}
36. Wright AL, Schwindt LA, Bassford TL, Reyna VF, Shisslak PAS
Catherine M amd Germain, Reed KL. Gender differences in academic
advancement: Patterns, causes, and potential solutions in one U.S.
college of medicine. Social Forces. 2003;68: 1297--1316.

\hypertarget{ref-bleidorn2016age}{}
37. Bleidorn W, Arslan RC, Denissen JJ, Rentfrow PJ, Gebauer JE, Potter
J, et al. Age and gender differences in self-esteem -- a cross-cultural
window. Journal of Personality and Social Psychology. 2016;111: 396.

\hypertarget{ref-jagsi_2016}{}
38. Jagsi R, Griffith KA, Jones R, Perumalswami CR, Ubel P, Stewart A.
Sexual harassment and discrimination experiences of academic medical
faculty. JAMA. 2016;315: 2120--2121.

\hypertarget{ref-Martin_2014}{}
39. Martin JL. Ten simple rules to achieve conference speaker gender
balance. PLoS computational biology. 2014;10: e1003903.

\hypertarget{ref-Tower_2007}{}
40. Tower G, Julie P, Brenda R. A multidisciplinary study of
gender-based research productivity in the world's best journals. Journal
of Diversity Management. 2007;2: 23--32.

\hypertarget{ref-Jordan_2008}{}
41. Jordan CE, Clark SJ, Vann CE. Do gender differences exist in the
publication productivity of accounting faculty?. Journal of Applied
Business Research. 2008;24: 77--85.

\hypertarget{ref-Britton_2000}{}
42. Britton DM. The epistemology of the gendered organization. Gender
and Society. 2000;14: 418--434.

\hypertarget{ref-Reagans_2001}{}
43. Reagans R, Zuckerman EW. Networks, diversity, and productivity: The
social capital of corporate R\&D teams. Organization Science. 2001;12:
502--517.

\hypertarget{ref-Hong_2004}{}
44. Hong L, Page SE. Groups of diverse problem solvers can outperform
groups of high-ability problem solvers. Proceedings of the National
Academy of Sciences. 2004;101: 16385--16389.

\hypertarget{ref-Whittington_2008}{}
45. Whittington KB, Smith-Doerr L. Women inventors in context:
Disparities in patenting across academia and industry. Gender \&
Society. 2008;22: 194--218.

\hypertarget{ref-Bear_2011}{}
46. Bear JB, Woolley AW. The role of gender in team collaboration and
performance. Interdisciplinary Science Reviews. 2011;36: 46--153.

\hypertarget{ref-Herrera_2012}{}
47. Herrera R, Duncan PA, Green MT, Skaggs SL. The effect of gender on
leadership and culture. Global Business and Organizational Excellence.
2012;31: 37--48.

\hypertarget{ref-Campbell_2013}{}
48. Campbell LG, Mehtani S, Dozier ME, Rinehart J. Gender-heterogeneous
working groups produce higher quality science. PloS ONE. 2013; e79147.

\hypertarget{ref-Ferber_1980}{}
49. Ferber MA, Teiman M. Are women economists at a disadvantage in
publishing journal articles? Eastern Economic Journal. 1980;6:
1189--193.

\hypertarget{ref-McDowell_1992}{}
50. McDowell JM, Smith JK. The effect of gender-sorting on propensity to
coauthor: Implications for academic promotion. Economic Inquiry.
1992;30: 68--82.

\hypertarget{ref-Ghiasi_2015}{}
51. Ghiasi G, Larivière V, Sugimoto CR. On the compliance of women
engineers with a gendered scientific system. PloS ONE. 2015;10:
e0145931.

\hypertarget{ref-Crow_2015}{}
52. Crow MS, Smykla JO. An examination of author characteristics in
national and regional criminology and criminal justice journals,
2008-2010: Are female scholars changing the nature of publishing in
criminology and criminal justice? American Journal of Criminal Justice.
2015;40: 441--455.

\hypertarget{ref-Fahmy_2017}{}
53. Fahmy C, Young JT. Gender inequality and knowledge production in
criminology and criminal justice. Journal of Criminal Justice Education.
2017;28: 285--305.

\hypertarget{ref-Zettler_2017}{}
54. Zettler HR, Stephanie M Cardwell, Jessica M C. The gendering effects
of co-authorship in criminology \& criminal justice research. Criminal
Justice Studies. 2017;30: 30--44.

\hypertarget{ref-Jadidi_2017}{}
55. Jadidi M, Karimi F, Lietz H, Wagner C. Gender disparities in
science? Dropout, productivity, collaborations and success of male and
female computer scientists. Advances in Complex Systems. 2017; 1750011.

\hypertarget{ref-Teele_2017}{}
56. Teele DL, Kathleen T. Gender in the journals: Publication patterns
in political science. PS: Political Science \& Politics. 2017;50:
433--447.

\hypertarget{ref-Araujo_2017a}{}
57. Araújo T, Elsa F. The specific shapes of gender imbalance in
scientific authorships: a network approach. Journal of Informetrics.
2017;11: 88--102.

\hypertarget{ref-Araujo_2017b}{}
58. Araújo T, Elsa F. Big Missing Data: are scientific memes inherited
differently from gendered authorship? arXiv preprint arXiv. 2017;
1706.05156.

\hypertarget{ref-Wahlund_1928}{}
59. Wahlund S. Zusammensetzung von populationen und
korrelationserscheinungen vom standpunkt der vererbungslehre aus
betrachtet. Hereditas. 1928;11: 65--106.

\hypertarget{ref-bergstrom_2016}{}
60. Bergstrom T, Bergstrom C, King M, Jacquet J, West J, Correll S. A
note on measuring gender homophily among scholarly authors.
http://eigenfactororg/gender/ assortativity/measuring\_homophilypdf.
2016;

\hypertarget{ref-macaluso_2016}{}
61. Macaluso B, Larivière V, Sugimoto T, Sugimoto CR. Is science built
on the shoulders of women? A study of gender differences in
contributorship. Academic Medicine. 2016;91: 1136--1142.

\hypertarget{ref-Bentley_2003}{}
62. Bentley JT, Adamson R. Gender differences in the careers of academic
scientists and engineers: A literature review. Special Report. 2003;

\hypertarget{ref-Long_2015}{}
63. Long MT, Leszczynski A, Thompson KD, Wasan SK, Calderwood AH. Female
authorship in major academic gastroenterology journals: A look over 20
years. Gastrointestinal Endoscopy. 2015;81: 1440--1447.

\hypertarget{ref-Bendels_2018}{}
64. Bendels MH, Bauer J, Schöffel N, Groneberg DA. The gender gap in
schizophrenia research. Schizophrenia Research. 2018;193: 445--446.

\hypertarget{ref-McKenzie_2017}{}
65. McKenzie K, Ramonas M, Patlas M, Katz DS. Assessing the gap in
female authorship in the journal emergency radiology: Trends over a
20-year period. Emergency Radiology. 2017;24: 641--644.

\hypertarget{ref-sheltzer_2014}{}
66. Sheltzer JM, Smith JC. Elite male faculty in the life sciences
employ fewer women. Proceedings of the National Academy of Sciences.
2014;111: 10107--10112.

\hypertarget{ref-Garfield_2006}{}
67. Garfield E. The history and meaning of the journal impact factor.
JAMA. 2006;295: 90--93.

\hypertarget{ref-nittrouer_2018}{}
68. Nittrouer CL, Hebl MR, Ashburn-Nardo L, Trump-Steele RC, Lane DM,
Valian V. Gender disparities in colloquium speakers at top universities.
Proceedings of the National Academy of Sciences. 2018;115: 104--108.

\hypertarget{ref-debarre_2018}{}
69. Débarre F, Rode N, Ugelvig L. Gender equity at scientific events.
Evolution Letters. 2018;in press: doi:10.1002/evl3.49.

\hypertarget{ref-wright_1949}{}
70. Wright S. The genetical structure of populations. Annals of Human
Genetics. 1949;15: 323--354.

\hypertarget{ref-wang_2016}{}
71. Wang YS, Erosheva EA. On the relationship between set-based and
network-based measures of gender homophily in scholarly publications.
arXiv preprint arXiv:161009026. 2016;

\hypertarget{ref-Benjamini_1995}{}
72. Benjamini Y, Hochberg Y. Controlling the false discovery rate: a
practical and powerful approach to multiple testing. Journal of the
Royal Statistical Society: Series B. 1995; 289--300.

\hypertarget{ref-Bonham_2017}{}
73. Bonham KS, Stefan MI. Women are underrepresented in computational
biology: An analysis of the scholarly literature in biology, computer
science and computational biology. PLoS Computational Biology. 2017;13:
e1005134.

\hypertarget{ref-Wren_2007}{}
74. Wren JD, Kozak KZ, Johnson KR, Deakyne SJ, Schilling LM, Dellavalle
RP. The write position: A survey of perceived contributions to papers
based on byline position and number of authors. EMBO reports. 2007;8:
988--991.

\hypertarget{ref-burkner_2016}{}
75. Bürkner P-C. brms: An R package for Bayesian multilevel models using
Stan. Journal of Statistical Software. 2016;80: 1--28.

\newpage

\section{Supporting information}\label{supporting-information}

\subsection{Supplementary figures}\label{supplementary-figures}

S1 Fig. Plot showing the percentage of papers that have 1, 2, 3, 4, or
\({\ge}5\) authors for each discipline in the dataset of Holman et al.
(2018). This information can also be found in S3 Data.

S2 Fig. Histogram showing the distribution of differences in \(\alpha'\)
between the 2015-16 and 2005-6 samples, where positive numbers indicate
an increase in \(\alpha'\) with time. The mean is slightly positive
(i.e.~0.004), indicating a mild increase in average \(\alpha'\) with
time.

S3 Fig. Histogram showing the difference between \(\alpha'\) calculated
for first and last authors. Positive values mean that \(\alpha'\) was
higher when calculated for first authors, and negative values mean
\(\alpha'\) was higher when calculated for last authors. The mean is
very slightly higher than zero, indicating that \(\alpha'\) tends to be
higher for first authors.

S4 Fig. Histogram of \(\alpha'\) for 325 unique combinations of journal
and country, using data from August 2015 - August 2016. The white areas
denote combinations for which \(\alpha'\) differs significantly from
zero (p \textless{} 0.05, following false discovery rate correction).

S5 Fig. Plot showing the 68 combinations of journal and author country
of affiliation for which \(\alpha'\) is significantly higher than
expected.

S6 Fig. Histogram showing the estimated degree to which \(\alpha'\) is
inflated by inter-country differences in author gender ratio, across the
285 journals for which we had adequate data after restricting the
analysis by country. The average inflation in \(\alpha'\) is negligible,
suggesting that Wahlund effects resulting from inter-country differences
have a neglible effect on our estimates of gender homophily.

S7 Fig. There is a very strong correlation between the values of
\(\alpha\) and \(\alpha'\) calculated for each journal, though in a
handful of cases the difference is considerable. The deviation between
\(\alpha\) and \(\alpha'\) is greatest for journals for which there is a
small sample size (see S8 Fig).

S8 Fig. For journals for which we recovered a small number of papers
(\textless{}100), the unadjusted metric \(\alpha\) was downwardly
biased. This fits our expectations: because authors cannot be their own
co-authors, small datasets will tend to produce negative estimates of
\(\alpha\) even if authors assort randomly with respect to gender (see
main text). This suggests that \(\alpha'\) is a more useful measure of
homophily and heterophily, especially for small samples.

\subsection{Supplementary tables}\label{supplementary-tables}

S1 Table. Sample sizes for the two datasets, which comprise papers
published in the timeframes August 2005 - August 2006, and August 2015 -
August 2016.

S2 Table. Number of journals showing statistically significant homophily
or heterophily, in two one-year periods. The significance threshold was
p \textless{} 0.05, and p-values were adjused using Benjamini-Hochberg
false discovery rate correction. Note that the power of our test is
lower for the 2005-2006 data because fewer papers were recovered per
journal: thus, it is not meaningful to compare the \% significant
journals (i.e.~11\% vs 24\%) between the two time periods.

\subsection{Supplementary datasets}\label{supplementary-datasets}

S1 Data: This spreadsheet shows the \(\alpha\) values calculated for
each journal, in the 2005 and 2015 samples, and for each type of author
(all authors, first authors, and last authors). The tables gives the
impact factor of each journal, the sample size, \(\alpha\) and
\(\alpha'\) and their 95\% CIs, and the p-value from a 2-tailed test
evaluating the null hypothesis that \(\alpha\) is zero (both raw and
FDR-corrected p-values are shown).

S2 Data: This file gives the number and percentage of paper that have 1,
2, 3, 4, or \({\ge}5\) authors for each \emph{journal} in the dataset of
Holman et al. (2018) \emph{PLoS Biology}. Note that the sample sizes
include papers for which the gender of one or more authors was not
determined by Holman et al.

S3 Data: This file gives the number and percentage of paper that have 1,
2, 3, 4, or \({\ge}5\) authors for each \emph{discipline} in the dataset
of Holman et al. (2018) \emph{PLoS Biology}. Note that the sample sizes
include papers for which the gender of one or more authors was not
determined by Holman et al.

S4 Data. The table shows the distribution of the \(\alpha'\) values
across journals, split by the research discipline. The gender ratio
column shows the percentage of women authors in the sample used to
calculate \(\alpha'\), across all authorship positions. In the last two
columns, the numbers outside parentheses give the number of journals
that deviate statistically significantly from zero, while the numbers
inside parentheses give the number that remain significant after false
discovery rate correction.


\end{document}
